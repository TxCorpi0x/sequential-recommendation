%% bare_jrnl.tex
%% V1.4b
%% 2015/08/26
%% by Michael Shell
%% see http://www.michaelshell.org/
%% for current contact information.
%%
%% This is a skeleton file demonstrating the use of IEEEtran.cls
%% (requires IEEEtran.cls version 1.8b or later) with an IEEE
%% journal paper.
%%
%% Support sites:
%% http://www.michaelshell.org/tex/ieeetran/
%% http://www.ctan.org/pkg/ieeetran
%% and
%% http://www.ieee.org/

%%*************************************************************************
%% Legal Notice:
%% This code is offered as-is without any warranty either expressed or
%% implied; without even the implied warranty of MERCHANTABILITY or
%% FITNESS FOR A PARTICULAR PURPOSE! 
%% User assumes all risk.
%% In no event shall the IEEE or any contributor to this code be liable for
%% any damages or losses, including, but not limited to, incidental,
%% consequential, or any other damages, resulting from the use or misuse
%% of any information contained here.
%%
%% All comments are the opinions of their respective authors and are not
%% necessarily endorsed by the IEEE.
%%
%% This work is distributed under the LaTeX Project Public License (LPPL)
%% ( http://www.latex-project.org/ ) version 1.3, and may be freely used,
%% distributed and modified. A copy of the LPPL, version 1.3, is included
%% in the base LaTeX documentation of all distributions of LaTeX released
%% 2003/12/01 or later.
%% Retain all contribution notices and credits.
%% ** Modified files should be clearly indicated as such, including  **
%% ** renaming them and changing author support contact information. **
%%*************************************************************************


% *** Authors should verify (and, if needed, correct) their LaTeX system  ***
% *** with the testflow diagnostic prior to trusting their LaTeX platform ***
% *** with production work. The IEEE's font choices and paper sizes can   ***
% *** trigger bugs that do not appear when using other class files.       ***                          ***
% The testflow support page is at:
% http://www.michaelshell.org/tex/testflow/



\documentclass[journal]{IEEEtran}
%
% If IEEEtran.cls has not been installed into the LaTeX system files,
% manually specify the path to it like:
% \documentclass[journal]{../sty/IEEEtran}

\usepackage[pdftex]{graphicx}
\graphicspath{{../pdf/}{../jpeg/}}
\DeclareGraphicsExtensions{.pdf,.jpeg,.png}




% Some very useful LaTeX packages include:
% (uncomment the ones you want to load)


% *** MISC UTILITY PACKAGES ***
%
%\usepackage{ifpdf}
% Heiko Oberdiek's ifpdf.sty is very useful if you need conditional
% compilation based on whether the output is pdf or dvi.
% usage:
% \ifpdf
%   % pdf code
% \else
%   % dvi code
% \fi
% The latest version of ifpdf.sty can be obtained from:
% http://www.ctan.org/pkg/ifpdf
% Also, note that IEEEtran.cls V1.7 and later provides a builtin
% \ifCLASSINFOpdf conditional that works the same way.
% When switching from latex to pdflatex and vice-versa, the compiler may
% have to be run twice to clear warning/error messages.






% *** CITATION PACKAGES ***
%
%\usepackage{cite}
% cite.sty was written by Donald Arseneau
% V1.6 and later of IEEEtran pre-defines the format of the cite.sty package
% \cite{} output to follow that of the IEEE. Loading the cite package will
% result in citation numbers being automatically sorted and properly
% "compressed/ranged". e.g., [1], [9], [2], [7], [5], [6] without using
% cite.sty will become [1], [2], [5]--[7], [9] using cite.sty. cite.sty's
% \cite will automatically add leading space, if needed. Use cite.sty's
% noadjust option (cite.sty V3.8 and later) if you want to turn this off
% such as if a citation ever needs to be enclosed in parenthesis.
% cite.sty is already installed on most LaTeX systems. Be sure and use
% version 5.0 (2009-03-20) and later if using hyperref.sty.
% The latest version can be obtained at:
% http://www.ctan.org/pkg/cite
% The documentation is contained in the cite.sty file itself.






% *** GRAPHICS RELATED PACKAGES ***
%
\ifCLASSINFOpdf
  % \usepackage[pdftex]{graphicx}
  % declare the path(s) where your graphic files are
  % \graphicspath{{../pdf/}{../jpeg/}}
  % and their extensions so you won't have to specify these with
  % every instance of \includegraphics
  % \DeclareGraphicsExtensions{.pdf,.jpeg,.png}
\else
  % or other class option (dvipsone, dvipdf, if not using dvips). graphicx
  % will default to the driver specified in the system graphics.cfg if no
  % driver is specified.
  % \usepackage[dvips]{graphicx}
  % declare the path(s) where your graphic files are
  % \graphicspath{{../eps/}}
  % and their extensions so you won't have to specify these with
  % every instance of \includegraphics
  % \DeclareGraphicsExtensions{.eps}
\fi
% graphicx was written by David Carlisle and Sebastian Rahtz. It is
% required if you want graphics, photos, etc. graphicx.sty is already
% installed on most LaTeX systems. The latest version and documentation
% can be obtained at: 
% http://www.ctan.org/pkg/graphicx
% Another good source of documentation is "Using Imported Graphics in
% LaTeX2e" by Keith Reckdahl which can be found at:
% http://www.ctan.org/pkg/epslatex
%
% latex, and pdflatex in dvi mode, support graphics in encapsulated
% postscript (.eps) format. pdflatex in pdf mode supports graphics
% in .pdf, .jpeg, .png and .mps (metapost) formats. Users should ensure
% that all non-photo figures use a vector format (.eps, .pdf, .mps) and
% not a bitmapped formats (.jpeg, .png). The IEEE frowns on bitmapped formats
% which can result in "jaggedy"/blurry rendering of lines and letters as
% well as large increases in file sizes.
%
% You can find documentation about the pdfTeX application at:
% http://www.tug.org/applications/pdftex





% *** MATH PACKAGES ***
%
%\usepackage{amsmath}
% A popular package from the American Mathematical Society that provides
% many useful and powerful commands for dealing with mathematics.
%
% Note that the amsmath package sets \interdisplaylinepenalty to 10000
% thus preventing page breaks from occurring within multiline equations. Use:
%\interdisplaylinepenalty=2500
% after loading amsmath to restore such page breaks as IEEEtran.cls normally
% does. amsmath.sty is already installed on most LaTeX systems. The latest
% version and documentation can be obtained at:
% http://www.ctan.org/pkg/amsmath





% *** SPECIALIZED LIST PACKAGES ***
%
%\usepackage{algorithmic}
% algorithmic.sty was written by Peter Williams and Rogerio Brito.
% This package provides an algorithmic environment fo describing algorithms.
% You can use the algorithmic environment in-text or within a figure
% environment to provide for a floating algorithm. Do NOT use the algorithm
% floating environment provided by algorithm.sty (by the same authors) or
% algorithm2e.sty (by Christophe Fiorio) as the IEEE does not use dedicated
% algorithm float types and packages that provide these will not provide
% correct IEEE style captions. The latest version and documentation of
% algorithmic.sty can be obtained at:
% http://www.ctan.org/pkg/algorithms
% Also of interest may be the (relatively newer and more customizable)
% algorithmicx.sty package by Szasz Janos:
% http://www.ctan.org/pkg/algorithmicx




% *** ALIGNMENT PACKAGES ***
%
%\usepackage{array}
% Frank Mittelbach's and David Carlisle's array.sty patches and improves
% the standard LaTeX2e array and tabular environments to provide better
% appearance and additional user controls. As the default LaTeX2e table
% generation code is lacking to the point of almost being broken with
% respect to the quality of the end results, all users are strongly
% advised to use an enhanced (at the very least that provided by array.sty)
% set of table tools. array.sty is already installed on most systems. The
% latest version and documentation can be obtained at:
% http://www.ctan.org/pkg/array


% IEEEtran contains the IEEEeqnarray family of commands that can be used to
% generate multiline equations as well as matrices, tables, etc., of high
% quality.




% *** SUBFIGURE PACKAGES ***
%\ifCLASSOPTIONcompsoc
%  \usepackage[caption=false,font=normalsize,labelfont=sf,textfont=sf]{subfig}
%\else
%  \usepackage[caption=false,font=footnotesize]{subfig}
%\fi
% subfig.sty, written by Steven Douglas Cochran, is the modern replacement
% for subfigure.sty, the latter of which is no longer maintained and is
% incompatible with some LaTeX packages including fixltx2e. However,
% subfig.sty requires and automatically loads Axel Sommerfeldt's caption.sty
% which will override IEEEtran.cls' handling of captions and this will result
% in non-IEEE style figure/table captions. To prevent this problem, be sure
% and invoke subfig.sty's "caption=false" package option (available since
% subfig.sty version 1.3, 2005/06/28) as this is will preserve IEEEtran.cls
% handling of captions.
% Note that the Computer Society format requires a larger sans serif font
% than the serif footnote size font used in traditional IEEE formatting
% and thus the need to invoke different subfig.sty package options depending
% on whether compsoc mode has been enabled.
%
% The latest version and documentation of subfig.sty can be obtained at:
% http://www.ctan.org/pkg/subfig




% *** FLOAT PACKAGES ***
%
%\usepackage{fixltx2e}
% fixltx2e, the successor to the earlier fix2col.sty, was written by
% Frank Mittelbach and David Carlisle. This package corrects a few problems
% in the LaTeX2e kernel, the most notable of which is that in current
% LaTeX2e releases, the ordering of single and double column floats is not
% guaranteed to be preserved. Thus, an unpatched LaTeX2e can allow a
% single column figure to be placed prior to an earlier double column
% figure.
% Be aware that LaTeX2e kernels dated 2015 and later have fixltx2e.sty's
% corrections already built into the system in which case a warning will
% be issued if an attempt is made to load fixltx2e.sty as it is no longer
% needed.
% The latest version and documentation can be found at:
% http://www.ctan.org/pkg/fixltx2e


%\usepackage{stfloats}
% stfloats.sty was written by Sigitas Tolusis. This package gives LaTeX2e
% the ability to do double column floats at the bottom of the page as well
% as the top. (e.g., "\begin{figure*}[!b]" is not normally possible in
% LaTeX2e). It also provides a command:
%\fnbelowfloat
% to enable the placement of footnotes below bottom floats (the standard
% LaTeX2e kernel puts them above bottom floats). This is an invasive package
% which rewrites many portions of the LaTeX2e float routines. It may not work
% with other packages that modify the LaTeX2e float routines. The latest
% version and documentation can be obtained at:
% http://www.ctan.org/pkg/stfloats
% Do not use the stfloats baselinefloat ability as the IEEE does not allow
% \baselineskip to stretch. Authors submitting work to the IEEE should note
% that the IEEE rarely uses double column equations and that authors should try
% to avoid such use. Do not be tempted to use the cuted.sty or midfloat.sty
% packages (also by Sigitas Tolusis) as the IEEE does not format its papers in
% such ways.
% Do not attempt to use stfloats with fixltx2e as they are incompatible.
% Instead, use Morten Hogholm'a dblfloatfix which combines the features
% of both fixltx2e and stfloats:
%
% \usepackage{dblfloatfix}
% The latest version can be found at:
% http://www.ctan.org/pkg/dblfloatfix




%\ifCLASSOPTIONcaptionsoff
%  \usepackage[nomarkers]{endfloat}
% \let\MYoriglatexcaption\caption
% \renewcommand{\caption}[2][\relax]{\MYoriglatexcaption[#2]{#2}}
%\fi
% endfloat.sty was written by James Darrell McCauley, Jeff Goldberg and 
% Axel Sommerfeldt. This package may be useful when used in conjunction with 
% IEEEtran.cls'  captionsoff option. Some IEEE journals/societies require that
% submissions have lists of figures/tables at the end of the paper and that
% figures/tables without any captions are placed on a page by themselves at
% the end of the document. If needed, the draftcls IEEEtran class option or
% \CLASSINPUTbaselinestretch interface can be used to increase the line
% spacing as well. Be sure and use the nomarkers option of endfloat to
% prevent endfloat from "marking" where the figures would have been placed
% in the text. The two hack lines of code above are a slight modification of
% that suggested by in the endfloat docs (section 8.4.1) to ensure that
% the full captions always appear in the list of figures/tables - even if
% the user used the short optional argument of \caption[]{}.
% IEEE papers do not typically make use of \caption[]'s optional argument,
% so this should not be an issue. A similar trick can be used to disable
% captions of packages such as subfig.sty that lack options to turn off
% the subcaptions:
% For subfig.sty:
% \let\MYorigsubfloat\subfloat
% \renewcommand{\subfloat}[2][\relax]{\MYorigsubfloat[]{#2}}
% However, the above trick will not work if both optional arguments of
% the \subfloat command are used. Furthermore, there needs to be a
% description of each subfigure *somewhere* and endfloat does not add
% subfigure captions to its list of figures. Thus, the best approach is to
% avoid the use of subfigure captions (many IEEE journals avoid them anyway)
% and instead reference/explain all the subfigures within the main caption.
% The latest version of endfloat.sty and its documentation can obtained at:
% http://www.ctan.org/pkg/endfloat
%
% The IEEEtran \ifCLASSOPTIONcaptionsoff conditional can also be used
% later in the document, say, to conditionally put the References on a 
% page by themselves.




% *** PDF, URL AND HYPERLINK PACKAGES ***
%
%\usepackage{url}
% url.sty was written by Donald Arseneau. It provides better support for
% handling and breaking URLs. url.sty is already installed on most LaTeX
% systems. The latest version and documentation can be obtained at:
% http://www.ctan.org/pkg/url
% Basically, \url{my_url_here}.




% *** Do not adjust lengths that control margins, column widths, etc. ***
% *** Do not use packages that alter fonts (such as pslatex).         ***
% There should be no need to do such things with IEEEtran.cls V1.6 and later.
% (Unless specifically asked to do so by the journal or conference you plan
% to submit to, of course. )


% correct bad hyphenation here
\hyphenation{op-tical net-works semi-conduc-tor}


\begin{document}
\bstctlcite{IEEEexample:BSTcontrol}
%
% paper title
% Titles are generally capitalized except for words such as a, an, and, as,
% at, but, by, for, in, nor, of, on, or, the, to and up, which are usually
% not capitalized unless they are the first or last word of the title.
% Linebreaks \\ can be used within to get better formatting as desired.
% Do not put math or special symbols in the title.
\title{Sequential Recommender Systems Deep Graph Evaluation and Denoising}
%
%
% author names and IEEE memberships
% note positions of commas and nonbreaking spaces ( ~ ) LaTeX will not break
% a structure at a ~ so this keeps an author's name from being broken across
% two lines.
% use \thanks{} to gain access to the first footnote area
% a separate \thanks must be used for each paragraph as LaTeX2e's \thanks
% was not built to handle multiple paragraphs
%

\author{MEHDI VALINEJAD,~\IEEEmembership{Student}% <-this % stops a space
% \thanks{M. Shell was with the Department
% of Electrical and Computer Engineering, Georgia Institute of Technology, Atlanta,
% GA, 30332 USA e-mail: (see http://www.michaelshell.org/contact.html).}% <-this % stops a space
% \thanks{J. Doe and J. Doe are with Anonymous University.}% <-this % stops a space
% \thanks{Manuscript received April 19, 2005; revised August 26, 2015.}
}

% note the % following the last \IEEEmembership and also \thanks - 
% these prevent an unwanted space from occurring between the last author name
% and the end of the author line. i.e., if you had this:
% 
% \author{....lastname \thanks{...} \thanks{...} }
%                     ^------------^------------^----Do not want these spaces!
%
% a space would be appended to the last name and could cause every name on that
% line to be shifted left slightly. This is one of those "LaTeX things". For
% instance, "\textbf{A} \textbf{B}" will typeset as "A B" not "AB". To get
% "AB" then you have to do: "\textbf{A}\textbf{B}"
% \thanks is no different in this regard, so shield the last } of each \thanks
% that ends a line with a % and do not let a space in before the next \thanks.
% Spaces after \IEEEmembership other than the last one are OK (and needed) as
% you are supposed to have spaces between the names. For what it is worth,
% this is a minor point as most people would not even notice if the said evil
% space somehow managed to creep in.



% The paper headers
\markboth{Sequential Recommender Systems Deep Graph Evaluation and Denoising, VOL.~01, NO.~01, DECEMBER~2023}%
{Shell \MakeLowercase{\textit{et al.}}: Sequential Recommender Systems Deep Graph Evaluation and Denoising}
% The only time the second header will appear is for the odd numbered pages
% after the title page when using the twoside option.
% 
% *** Note that you probably will NOT want to include the author's ***
% *** name in the headers of peer review papers.                   ***
% You can use \ifCLASSOPTIONpeerreview for conditional compilation here if
% you desire.




% If you want to put a publisher's ID mark on the page you can do it like
% this:
%\IEEEpubid{0000--0000/00\$00.00~\copyright~2015 IEEE}
% Remember, if you use this you must call \IEEEpubidadjcol in the second
% column for its text to clear the IEEEpubid mark.



% use for special paper notices
%\IEEEspecialpapernotice{(Invited Paper)}




% make the title area
\maketitle

% As a general rule, do not put math, special symbols or citations
% in the abstract or keywords.
\begin{abstract}
Recent advancements in sequential recommender systems (SRSs) aim to provide users with better, personalized 
recommendations. While traditional models like collaborative filtering have been effective, SRSs focus on 
understanding user behaviors over time. However, SRSs, especially those using Graph Convolutional Neural Networks (GCNNs), 
face challenges such as complexity and inefficiency. This research compares traditional GCNNs with Graph Neighborhood 
Filters (GNFs), specifically neighborhood graph filters (NGFs). NGFs address challenges in GCNNs by providing a stable 
foundation using k-hop neighborhood adjacency matrices. The goal is to show the benefits of NGFs over traditional 
filters in practical applications like graph signal denoising and node classification. The study explores existing 
challenges in SRSs and contributes insights to improve recommendation systems. The research commits to evaluating NGFs' 
denoising performance against current methods, aiming to demonstrate their superiority in accuracy, scalability, 
and efficiency for sequential recommender systems. This is the GitHub repository for this report and implementation 
https://github.com/TxCorpi0x/sequential-recommendation
\end{abstract}

% Note that keywords are not normally used for peerreview papers.
\begin{IEEEkeywords}
Recommender Systems, GCNNs, NGFs.
\end{IEEEkeywords}






% For peer review papers, you can put extra information on the cover
% page as needed:
% \ifCLASSOPTIONpeerreview
% \begin{center} \bfseries EDICS Category: 3-BBND \end{center}
% \fi
%
% For peerreview papers, this IEEEtran command inserts a page break and
% creates the second title. It will be ignored for other modes.
\IEEEpeerreviewmaketitle



% The very first letter is a 2 line initial drop letter followed
% by the rest of the first word in caps.
% 
% form to use if the first word consists of a single letter:
% \IEEEPARstart{A}{demo} file is ....
% 
% form to use if you need the single drop letter followed by
% normal text (unknown if ever used by the IEEE):
% \IEEEPARstart{A}{}demo file is ....
% 
% Some journals put the first two words in caps:
% \IEEEPARstart{T}{his demo} file is ....
% 
% Here we have the typical use of a "T" for an initial drop letter
% and "HIS" in caps to complete the first word.


% === I. Introduction ========================
% =================================================================================
\section{Introduction}

\IEEEPARstart{T}{he} field of sequential recommender systems (SRSs) has witnessed significant advancements 
in recent years, driven by the growing need to provide more accurate, personalized, and timely recommendations 
to users. While traditional recommendation models, such as collaborative filtering and content-based filtering, 
have proven effective, the emergence of SRSs addresses the limitations of these approaches by focusing on 
understanding and modeling the sequential user behaviors, interactions, and the evolution of user preferences 
over time. Despite their success, many SRSs, particularly those based on Graph Convolutional Neural Networks 
(GCNNs), face challenges related to computational efficiency, model complexity, and susceptibility to 
over-parameterization.

In this context, this research plans to build a comparison context for SRSs conventional GCNNs denoising
with Graph Neighborhood Filters (GNFs). The motivation for this effort is rooted in the observed challenges 
of traditional GCNNs, such as numerical errors, limitations in practical depth, and sensitivity to graph 
topology variations \cite{tenorio2021robust}. The proposed GNFs, referred to as neighborhood graph filters (NGFs), present a family 
of graph filters that leverage k-hop neighborhood adjacency matrices, providing a more robust and numerically 
stable foundation for designing deep neighborhood GCNNs.

The primary goal of this research is to demonstrate the advantages of employing NGFs over classical graph filters 
in practical applications. To showcase the effectiveness of this approach, the study employs NGFs in the design 
of deep neighborhood GCNNs and evaluates their performance in graph signal denoising \cite{tenorio2021robust} \cite{FENG2022113} 
and node classification tasks using both synthetic and real-world datasets.

The subsequent sections of this research delve into the challenges faced by existing SRSs, particularly those 
utilizing GCNNs, and present a comprehensive exploration of the proposed GNF-based approach. The experimental 
results, supported by evaluations on various datasets, highlight the superior performance of the new model in 
terms of robustness, efficiency, and computational scalability. By addressing these key issues in SRSs, the 
research contributes to the ongoing evolution of recommendation systems, paving the way for more effective and 
scalable solutions in the era of dynamic user preferences and evolving content landscapes.


% === II. Literature Review ========================
% =================================================================================
\section{Literature Review}
Sequential recommender systems (SRSs) have evolved as a crucial component in addressing the dynamic nature \cite{10.1145/3383313.3412258} of user 
preferences and the temporal aspect of interactions. Traditional recommendation models, such as collaborative filtering 
and content-based filtering, laid the foundation for personalized suggestions. However, the emerging landscape of SRSs, 
particularly those employing Graph Convolutional Neural Networks (GCNNs), has faced challenges that prompt researchers 
to explore novel alternatives.

\subsection*{1. Temporal Dynamics and Personalization}
The importance of considering temporal information in recommendation systems is highlighted by various works. The SASRec model, 
inspired by the Transformer architecture, has demonstrated state-of-the-art results by leveraging attention mechanisms. 
However, limitations arise in terms of personalization, prompting the introduction of personalized models like SSE-PT, 
which outperforms SASRec by incorporating personalized user embeddings.


% \begin{figure}[!t]
%   \centering
%   \includegraphics[width=3.5in]{photo/hand-landmarks.png}
%   The list of detected point of the hand by landmark hand detector.
%   \caption{Landmark hand gesture detected joints}
%   \label{fig_sim}
% \end{figure}

\subsection*{2. Multi-Behavior}

The exploration of dependencies between different user behaviors is addressed by models like MB-AGCN \cite{PENG2023111040}. This model focuses on 
personalized interaction patterns and cross-typed behavioral interdependencies, showcasing improvements over existing methods 
in multi-behavior recommendation scenarios.

\subsection*{3. Graph Neural Networks}

The SURGE model \cite{10.1145/3404835.3462968} introduces a graph neural network approach to sequential recommendation, addressing challenges associated with 
implicit and noisy user behavior signals. The model leverages metric learning to cluster user preferences, demonstrating significant 
performance gains over state-of-the-art methods.

\subsection*{4. SR in Real-Time}

Acknowledging the timeliness and contextual accuracy required in contemporary digital marketing, sequential recommendation systems 
based on autoencoders and GRU models offer promising results. Real-time predictions \cite{SRMNPP} in a production dataset of credit card transactions 
showcase the potential of sequential recommendation systems.

\subsection*{5. Hybrid}

Hybrid recommender systems \cite{SRMNPP,9207880}, like U2CMS, aim to enhance personalized content recommendations by unifying similarity models, 
collaborative, and content-based approaches with Markov chains for sequential recommendation. The model proves effective in handling 
sparsity issues and outperforms existing state-of-the-art recommendation systems.

\subsection*{6. Complexities and Challenges}

A systematic review on sequential recommender systems (SRSs) emphasizes the unique characteristics and challenges \cite{Wang_2019} in this research 
domain. The research categorizes key challenges, presents recent developments, and outlines important research directions in the 
evolving field of SRSs.

\subsection*{7. Attention Mechanisms and Over-Parameterization}

Sequential recommender systems based on attention mechanisms \cite{9123874}, including RNNs, CNNs, memory networks, and attention networks, have 
gained popularity for their ability to capture dynamic preferences. However, the complexity and over-parameterization issues, 
particularly in Transformer models, pose challenges for effective deployment under limited resources.

\subsection*{8. Temporal Graph Networks for Cold-Start Problem}

Temporal graph networks emerge as powerful tools for addressing the cold-start problem in sequential recommender systems. The 
proposed exploration method using Rooted PageRank and bipartite graphs aims to mitigate feedback loops and data distribution shifts, 
providing competitive results on popular benchmarks \cite{9963919}.

\subsection*{9. Deep Attention-based Sequential Models}

The DAS model tackles the challenges of modeling both short-term and long-term preferences in user-item interaction sequences. 
By incorporating an embedding block, an attention block, and a fully-connected block, the model demonstrates superiority over 
state-of-the-art approaches in SRSs through extensive experiments \cite{9123874}.

\subsection*{10. Challenges in Graph Convolutional Neural Networks (GCNNs)}

Classical graph filters in GCNNs are prone to numerical errors and limitations in practical depth. The proposed NGFs, 
a form of graph neighborhood filters, address these challenges by leveraging k-hop neighborhood adjacency matrices, offering a 
more robust foundation for designing deep neighborhood GCNNs.

\subsection*{11. Transformer Models in Recommender Systems}

Transformer models have become the backbone of many SRSs due to their ability to capture long-range dependencies and sequential 
patterns. However, the growth in model size raises concerns about computational efficiency, prompting researchers to explore 
compression techniques for these architectures \cite{LI2024122260}.


% \begin{figure}[!t]
%   \centering
%   \includegraphics[width=1.5in]{photo/aruco-markers.png}
%   Sample chart of ArUco Markers.
%   \caption{ArUco Markers}
%   \label{fig_sim}
% \end{figure}


% === III. Contribution of Study ========================
% =================================================================================

\section{Contribution of Study}

In response to the challenges and complexities outlined in the existing literature on sequential recommender systems (SRSs), 
particularly those employing Graph Convolutional Neural Networks (GCNNs), this study aims to make a contribution by 
evaluating and tuning the denoising of Graph Neighborhood Filters (GNFs) \cite{tenorio2021robust} \cite{FENG2022113} to evaluate or 
improve the overall efficacy of SRSs. The key contribution of this research lies in addressing the limitations associated with traditional 
graph filters and leveraging GNFs or alternative approach to enhance various aspects of sequential recommendations.

While GCNNs have proven effective in capturing graph-structured data, their generalization capabilities may be hindered by 
over-parameterization. Graph Neighborhood Filters are anticipated to contribute to improved generalization by providing a more 
parsimonious representation of graph relationships. This, in turn, enhances the model's adaptability to diverse datasets and 
recommendation scenarios.

The study plans to conduct a comprehensive evaluation and improve denoising performance of the proposed Graph Neighborhood 
Filters by benchmarking them against state-of-the-art methods on various real-world datasets. This thorough assessment aims 
to demonstrate the superiority of the proposed approach in terms of recommendation accuracy, scalability, and computational efficiency.



% === IV. Methodology ========================
% =================================================================================

\section{Methodology}

\subsection*{1. Data Collection}

\begin{itemize}
  \item \textbf{Datasets}: Select diverse real-world datasets representing denoising performance of the GNFs.
  \item \textbf{Data Preprocessing}: Clean and preprocess datasets, handling missing values, encoding categorical features, and ensuring a standardized format.
\end{itemize}

\subsection*{2. Graph Construction}

\begin{itemize}
  \item \textbf{User-Item Interaction Graph}: Build a graph representation of user-item interactions, considering temporal dynamics and denosing dependencies.
  \item \textbf{Graph Neighborhood Extraction}: Define k-hop neighborhood adjacency matrices to capture local graph structures.
\end{itemize}

\subsection*{3. Graph Neighborhood Filters Integration}

\begin{itemize}
  \item \textbf{Graph Neighborhood Filters (GNFs) Tuning}: Modify GNFs denosing mechanism and Hyper-Parameter manipulation.
  \item \textbf{Denosing Performance}: Optionally, explore graph noise filtering techniques to further enhance efficiency and scalability.
\end{itemize}

\subsection*{4. Model Architecture}

\begin{itemize}
  \item \textbf{Sequential Recommender System (SRS)}: Establish a baseline SRS model using traditional Graph Convolutional Neural Networks (GCNNs).
  \item \textbf{Graph Neighborhood Filters Integration}: Integrate the designed GNFs into the SRS architecture, replacing traditional graph filters with improved denoising capabilities.
\end{itemize}

\subsection*{5. Model Training}

\begin{itemize}
  \item \textbf{Loss Function}: Define appropriate loss functions for the SRS model with GNFs, considering the nature of the recommendation task (e.g., ranking loss for sequential recommendations).
  \item \textbf{Optimization}: Utilize optimization algorithms such as stochastic gradient descent to train the model.
\end{itemize}

\subsection*{6. Evaluation Metrics}

\begin{itemize}
  \item \textbf{Performance Metrics}: Employ standard evaluation metrics for recommender systems, including precision, recall, F1 score, and ranking metrics like NDCG@k.
  \item \textbf{Comparison}: Benchmark the performance of the enhanced SRS with GNFs against current implementation of GFNs \cite{tenorio2021robust}.
\end{itemize}

\subsection*{7. Experimental Setup}

\textbf{Baseline Comparison}: Include a comparison with traditional GCNN-based SRS to highlight the impact of GNFs and denoising mechanisms.

\subsection*{8. Open-Source Implementation}

\begin{itemize}
  \item \textbf{Code Repository}: Share an open-source implementation of the enhanced SRS with GNFs, including pre-processing scripts, model training, and evaluation code.
  \item \textbf{Documentation}: Provide comprehensive documentation for researchers and practitioners to reproduce the study's results and extend the proposed approach.
\end{itemize}


% === V. DATASETS ========================
% =================================================================================

\section{DATASETS}

\begin{itemize}
  \item \textbf{Movie Lens 1M}
  \item \textbf{IMDB}
\end{itemize}


% === VI. EXPERIMENTAL RESULTS ========================
% =================================================================================

\section{EXPERIMENTAL RESULTS}

\subsection*{Compression of Numerical Errors}

Classical graph filters (GFs) in Graph Convolutional Neural Networks (GCNNs) are prone to numerical errors, especially with high-order polynomials,
The application of Neighborhood Graph Filters (NGFs) effectively mitigates numerical errors, allowing for smoother and more accurate computations during the training and inference phases

\subsection*{Enhanced Robustness to Graph Topology Errors}

Traditional GFs face challenges in maintaining robustness to errors in the topology of the graph, leading to suboptimal performance in real-world scenarios.
NGFs demonstrate enhanced robustness to errors in the graph's topology. This improvement is crucial for applications in dynamic and evolving graph structures commonly encountered in real-world datasets.

\ifCLASSOPTIONcaptionsoff
  \newpage
\fi


\bibliographystyle{IEEEtran}
\bibliography{IEEEabrv,Bibliography}


\begin{IEEEbiographynophoto}{Mehdi Valinejad}
Received the B.S. degree in industrial engineering from the Azad University (South Tehran Branch) in 2012, and is currently working Master's. degree at the University of Bahcesehir at Istanbul.
\end{IEEEbiographynophoto}


\end{document}


